\documentclass{cugrep}

\title{中国地质大学研究生课程论文\TeX{}模板}
\classname{\TeX{}模板教程}
\college{计算机学院}
\major{计算机科学与技术}
\sno{20131001333}
\teacher{教师名字}
\class{硕士}
\author{Timozer}
\begin{document}

\maketitle
\pagenumbering{roman}
\tableofcontents
\thispagestyle{plain}
\clearpage
\pagenumbering{arabic}

\chapter{介绍}
这是中国地质大学 (武汉) 研究生课程论文的\TeX{}模板.

测试换行. 先来看看页面边距的是什么样子的, 怎么总是感觉有点儿窄呢. 测试换行, 测试换行, 测试换行, 测试换行.
你有没有通过自律去尝试改变自己?

比如开始坚持跑步,或者开始早起,希望从此晨钟暮鼓,不虚此生。

但是,我更关心的,你或者身边人的自律,为什么多数情况都难以坚持,最后就不了了之了呢?



当然,能够做到长期自律,确实太辛苦了。这意味着在减肥时,不能吃美味的甜食;冬天清早,要离开自己温暖的被窝。

后来你放弃了,放弃的开始是因为找到了某种合理的借口,比如跑步其实挺伤膝盖的,早起并不会给身体带来实质性好处。

你会认为自己自控力不够坚强,或是性格不够坚韧。扭头看看那些能够长期坚持的大神,想想还是认怂了,他们能做到高度自律,的确让人佩服啊。



但,真的是这样吗?

如果你进行深度思考,会发现,自律这件事,可能和你想象的完全不一样。



02 自律究竟意味着什么?
想想看,为什么你一定非得自律呢?自律究竟意味着什么?

你可能说,这不是废话么,自律是让自己变得更好。请注意,让自己变得更好,意味着你并不喜欢当下的状态,你希望离开它。

而这个需要离开和并不喜欢的状态,叫做“自我”。一个不自律的人,就会显得更“自我”,这就像筷子的两端,放弃掉这头,就会走向另一头。这个应该比较好理解。



很多时候,我们说那个人很”自我“,大致是指一个人显得自私、不顾未来、没有时间观念、也不太在乎别人的感受。我们想离开这种状态,倒也在情理之中。

但它和自律是正负两个极端,要更好理解自律,也需要去理解“自我”。

所以今天主要探讨“自律系统”和“自我系统”的问题,这是人类两个最重要的动力系统。我借用大家最熟悉的《西游记》来形象化,方便你的理解,主要分为三个部分:


1、“沙僧”,代表“自律系统”,自从皈依佛门,清规戒律一直做得非常不错。

2、“八戒”,代表“自我系统”,贪图享受的代表

3、“X”,隐藏人物,稍后便知。



03 “沙僧”-自律系统
自从皈依佛门,沙僧的清规戒律一直做得非常不错,给人的感觉是“很守规矩”。

但沙僧可能是你在《西游记》里最不了解的一个角色,它其实有很多隐藏因素。这暗示我们的自律系统也同样如此,我给你完整还原一下:


1、失去控制

正如你从来不注意饮食,然后终于生病了,是对“身体健康”这件事失去控制。自律的很多前提,都是对某个事件或整个人生失去控制。

比如你为什么要减肥?那是出于对“美”失去控制,也有可能是对“别人对于你现在肥胖身材评价”这件事失去控制。

你为什么会自律去背单词呢?那是因为词汇量不足,会限制你在英语学习上的控制感。



2、防御机制启动

当你失控时,大脑中的“岛叶”部分就变得异常活跃,因此产生的激素让我们感觉十分痛苦。

但你本能地就想消除这些痛苦啊,所以你的大脑会采取行动,这就是防御机制(Defense mechanisms)的启动,主要有两个方面:

第一个是压抑(Repression),这是去人性化

比如你喜欢一个女生,对她有很大冲动。但你只能表现得彬彬有礼,慢慢接触。这个过程就有很大的压抑。

再比如我前段时间痛风,大脚趾肿得像个熊掌一样,痛得死去活来。然后逐渐开始饮食清淡,其中也涉及到美味的压抑。

所有的自律都是去人性化的过程,比如你学会早起、戒赌、戒撸,再比如佛教里的“持戒”,朱熹的“存天理,灭人欲”。都是让我们摆脱人性,成为一种超越人性的理想化状态。

第二个是“升华”(sublimation),可以使人角色化

经常压抑自己的人性是为了什么呢?那是为了获得升华,也就是把不好的地方转化为外部认同。

而最好的外部认同,就是成为别人期待的角色。这点太重要了,事实上我们从小到大,大部分的精力就是在做这件事情。

比如你在高中时候的自律,是为了获得“名牌大学生”这个角色;在工作里能做到遵纪守法,很多时候是为了成为领导眼中的好员工;而许多女性,终期一生都是为了成为好太太、好母亲、好儿媳。



3、工业化人格

自律的本质是机械思维,希望用大量重复的活动,取得某些方面的成就。

这个大家应该深有体会,比如你不能看电视,而重复演算数学题;或者每天背多少单词,每天做多少试卷。

所以自律这件事并不神奇,我们每个人从小就是自律的高手,就是要被训练成工业化人格。



所以,“自律系统”的本质是去人性化,寻求外部认同(工业化人格)。

想想看,身边那些看上去极度自律的人(比如我),是不是都需要一个社交平台以求获得认同呢?如果完全没有人关注他们,那么这些自律的行为还能持续多久呢?


04 “八戒”-自我系统
看到这里,你可能不太认同,“没对啊,越是自律的高手,越不会在意别人的眼光啊?”

不着急,我们来看看人类的另一套动力系统,即“自我系统”,很像“八戒”这个角色。

\chapter{使用方法}

\end{document}
