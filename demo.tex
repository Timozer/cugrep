\documentclass{cugrep}

\title{中国地质大学研究生课程论文\TeX{}模板}
\classname{\TeX{}模板教程}
\college{计算机学院}
\major{计算机科学与技术}
\sno{20131001333}
\teacher{教师名字}
\class{硕士}
\author{Timozer}
\begin{document}

\maketitle
\frontmatter
\cugabstract{
     这是中国地质大学(武汉)研究生课程论文的\LaTeX{}模板的说明文档. 主要对模板进行了介绍, 安装, 使用等方面的说明, 此外还对模板的实现细节进行了一定的说明. 版本历史可以查看到每个版本实现了哪些功能.
 }
\cugkeywords{\TeX{}, \LaTeX{}, 模板, 报告}
\makeabstract 
\tableofcontents
\clearpage
\mainmatter

\chapter{介绍}
这份中国地质大学(武汉)研究生课程论文的 \LaTeX{} 模板是我根据中国地大学(武汉)研究生课程论文的 {\sc WORD} 版来编写的. 
值得一提的是, 在 {\sc Word} 版中只定义了封面的内容和下一页的评语表, 对其他格式并没有明确的要求, 因此, 
在这份 \LaTeX{} 模板中有关其他方面的格式都是没有参照标准的, 一切以我的审美为标准 :)~.

\section{这是中文小节测试}
{
\noindent
{\textbf{English Bold}}\\
{\bf English Bold Fonts test} \\
{\textit{English Italic Fonts test}}\\
{\sc EnglishSmallCapsFontsTest}}
\section{这是代码测试}

看看间距

\begin{tcodeenv}{cpp代码}{code:cpp}
    \begin{minted}[firstnumber=30]{cpp}
    int sum(int _l, int _r)
    {
        return _l + _r;
    }
    \end{minted}
\end{tcodeenv}

看看间距
\begin{minted}[firstnumber=10]{cpp}
    /* $ a + b = c$ 
        \LaTeX{}
    */
    int main(){
        return 0;
    }

    int abs(int a)
    {
        return a;
    }
\end{minted}

{\tt This is normal font size}, \tcodeinline{tex}{\LaTeX{}}

\chapter{安装}

\chapter{使用方法}

\chapter{实现细节}

\chapter{版本历史}

\section{Version 0.3}
\label{sec:version_0_3}
因为想不起版本0.1-0.2做了哪些功能, 所以在此就不介绍了.

这个版本主要是实现了中国地质大学(武汉)研究生课程论文 Word 模板中的封面和后面的评语表.
\section{Version 0.4}
\label{sec:version_0_4}

目前在开发中, 暂时实现了摘要的添加. 因为这个\TeX{}模板主要 使用的是
ctexbook 文档类, 所以没有摘要这一环境, 暂时添加了摘要环境. 我不是太
想使用摘要这个环境, 使用环境意味着摘要和关键词的格式需要自己调, 这
很不舒服.

2018年 1月26日 星期五 21时50分38秒 CST

我新定义了一个命令\verb|\makeabstract|, 像\verb|\maketitle|一样, 可以
使用这个命令来生成摘要页面. 在使用之前, 你必须给变量\verb|\cugabstract, \cugkeywords| 赋值. 也许我可以起更好听的名字:).

2018年 1月28日 星期日 17时01分20秒 CST

为了使行文方便, 我重新定义了\verb|\frontmatter, \mainmatter, \backmatter| 这四个命令. 

\begin{itemize}
    \item \verb|\frontmatter| 这个命令实现的功能是关闭 chapter 计数, 使用 roman 数字来标识页码;
    \item \verb|\mainmatter| 这个命令实现的功能是打开 chapter 计数, 使用 arabic 数字来标识页码;
    \item \verb|\backmatter| 这个命令实现的功能是关闭 chapter 计数.
\end{itemize}

折腾了一会儿页眉页脚, 包括首页页眉页脚. 但是成果比较感人, 首页之外的页面实现了页眉页脚的设置,
页眉左边为``中国地质大学(武汉)研究生课程论文'', 右边为节标题, 节标题的格式有点儿不好,需要改进.
另一个就是首页的页眉页脚设置好像有点儿问题, 只是让它在下面显示页码, 结果没有显示, 目前还不知道问题所在. 

2018年 1月29日 星期一 12时04分22秒 CST

今天终于查到了首页页脚没显示的问题. 在 \verb|\ctexset| 里面设置 \verb|chapter| 的时候设置了页
面的格式为 \verb|empty|, 把后面的设置冲掉了. 现在更改为 \verb|plain| 格式. 新的问题又来了, 首
页页眉会显示一条横线, 看着不舒服, 去掉吧. 

ctex 本身对 appendix 的设置就很不错了, 所以我这里就不对其重新定义了, 删除了原来对 \verb|\appendix|
的重定义. 

看了好多大学的论文模板, 发现附录一般都是放在最后的, 而参考文献, 致谢这些在附录的前面. 这就产生了一个问题.
在写文档的时候, 我们用 \verb|\appendix| 命令来重置 chapter 计数, 并使用字母来表示章号, 而使用 \verb|\backmatter| 
来关闭 chapter 计数, 从而使 \verb|\chapter| 命令产生的章标题(参考文献, 致谢等) 不包含章号. 这样无形中
的一个顺序使得结果和自己所想甚远. \verb|\backmatter| 命令关闭 chapter 计数, \verb|\appendix| 是不会打开的.
所以这里要重新定义 \verb|\appendix|, 使得它会打开 chapter 计数. 

0.4 版本开发搞一段落. 回顾一下该版本实现的功能:

\begin{itemize}
    \item 添加了摘要命令 \verb|\makeabstract|, 使得可以方便生成摘要;
    \item 定义了报告的写作顺序, \verb|\frontmatter, \mainmatter, \backmatter, \appendix|, 严格按照这个顺序来定义
        报告的结构, 如果不按此来写, 很有可能不是你想要的结果;
    \item 重定义了 \verb|\frontmatter, \mainmatter, \appendix| 等命令.
\end{itemize}

\section{Version 0.5}

2018年 1月30日 星期二 17时07分04秒 CST

今天正式开始开发 0.5 版本.

昨天在推送 0.4 版本后发现了一个问题, 字体配置有问题. 0.4 版本及之前的版本中, 字体配置是我
自己在网上下载的字体, 而且也并没有将其安装到系统中, 只是将其放到了一个文件夹中, 然后使用
fontspec 宏包中的 \verb|\defaultfontfeature| 命令指定其路径. 这就造成了只能使用该路径下的
字体的问题. 关于如何添加多个路径以及把 \LaTeX{} 默认搜索路径添加进来的操作, 我不会. 谷歌了
很久都没有查到, 所以就放弃了, 如果你们谁知道如何解决这个问题, 烦请联系我, 不胜感激. 

在放弃了那个方法后, 我就使用了 \LaTeX{} 的默认搜索路径, 并且把自己要使用的字体安装到系统中.
这样, 我就可以在 \LaTeX{} 中使用了. 

这个问题解决后, 突然发现又有新问题出现了 (程序开发中解决一个 BUG 又产生了一堆新的 BUG 的即
视感有木有 \verb|>_<| ). 那就是英文字体中的加粗, 斜体, 小型大写字母命令不起作用了, 无衬线和等宽字体
的命令还能骑作用, 不是太清楚什么问题, 在寻找中.

关于这个问题, 我也是谷歌了很久的, 烦. 资料出奇的少.

最终找到一个类似的问题, 他的解决方案是去掉了一个宏包, 然后就可以了. 我首先被这个人的想法
惊到了, 突然想起以前看过一篇文章, 讲得是一个不懂电脑的电脑维修人员维修电脑的方法, 他的修电脑
方法和这个人类似, 也是将电脑的零件各种拔插, 本来的开不了机的电脑就可以开机了 (ps. 难道这种
方法是计算机领域的一大特色\verb|^_^|).

我决定采用他这种方法. 首先不加入任何宏包, 看这是不是 ctexbook 文档类本身的问题 (ps. 很明显是
我想多了, 人家那么多大神在维护, 怎么可能出这种问题). 显然不是, 我又想会不会是我自己写的 
timozerfont.sty 这个文件出了问题 (ps. 才疏学浅, 出问题在所难免). 在测试文档里引入了我的宏包, 
测试通过, 并没有问题. 那么就是我引入的宏包之间的冲突了. 目前在排除中 \ldots .

在我将以下这些宏包关闭后, 字体正常了.

\begin{itemize}
    \item \{amstext, amsmath, amssymb, amsfonts, mathrsfs, bm, mathtools, newtxtext, courier\}
    \item \{graphicx, subcaption\}
    \item \{longtable, makecell, tabu, booktabs\}
    \item \{natbib\}
    \item \{timozercode\}, 这个是我自己写的一个宏包, 用来方便输入代码的, 具体用法应该在该模板大体完善的时候会写.
\end{itemize}


接下来一组一组的去开启这些宏包, 使用排除法来找出产生冲突的宏包. 

很幸运, 在我开启第一组的时候就发现那几个命令工作不正常了. 后面三组不用看了.

使用二分法, 经过几次测试后, 发现是 newtxtext 的问题. 

我又对该宏包进行了单独测试, 发现其和 ctexbook 文档类并不冲突, 奇哉怪哉. 难道
是和我写的 timozerfont.sty 宏包有冲突? 经过测试, 确实是和该宏包冲突, 在没有
找到解决方法之前, 我先把 newtxtext 宏包禁用了. 

2018年 2月 1日 星期四 10时28分29秒 CST

经过昨天的一番思想斗争以及查阅资料, 看宏包的说明文档, 我终于决定使用 minted 宏包来代替 lstlistings 宏包
来实现代码插入.

2018年 2月 1日 星期四 18时10分35秒 CST

在为代码添加背景色的时候, 发现 color 宏包不能正常使用, 换成了 xcolor 宏包. 

今天实现了行内代码命令的定义, 提供了一个代码浮动体环境,
并且对代码环境的外观进行了设置. 基本任务完成, 如果有空,
我会考虑添加从文件中导入代码的功能.


\backmatter
\chapter{参考文献}
\chapter{致谢}
\appendix
\chapter{测试附录}
这里是附录, 看看是否会显示正常.

\backmatter 
\end{document}
